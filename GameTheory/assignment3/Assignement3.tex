\documentclass{article}
\usepackage{blindtext}
\usepackage[utf8]{inputenc}

\usepackage{amsthm, amsmath, amssymb}
\usepackage{geometry, setspace, graphicx, enumerate}
\usepackage{listings}
\usepackage[usenames, dvipsnames]{color}
\usepackage{booktabs}

\DeclareMathOperator*{\argmax}{arg\,max}
\DeclareMathOperator*{\argmin}{arg\,min}

\newenvironment{answer}{\par\color{ForestGreen}}{\par}
\setlength\parindent{0pt}

\title{Assignment 3}
\author{Guoxin SUI}
\date{\today}

\begin{document}

\maketitle

\section{First Half}
\subsection{Do one of the followings:}
\begin{enumerate}
    \item
    Maximum Market Equilibrium Dynamics: Starting at truthful biddings,
    how would the player change their bids? Assuming best response, would the process converge?
    \item
    Prove or disprove that the maximum revenue market equilibrium is a symmetric Nash Equilibrium in GSP.
    \item
    What is the maximum revenue symmetric Nash equilibrium in GSP?
\end{enumerate}

\begin{answer}
Question 1: \\
The players will always try to increase his utility. There are two possibilities :
  \begin{itemize}
  \item give higher bid and get more quality ;
  \item give lower bid and pay less.
  \end{itemize}
  When there are more than one items of different qualities, the process doesn't converge.

  When there are only one item or all the items have the same quality, the process converge. For the market of N bidders and M items, the bids of the first M bidders converge to $v_{M+1}$
\end{answer}

\subsection{Do two of the followings:}
\begin{enumerate}
    \item
    How much one can gain from VCG to maximum VCG?
    \item
    Give Examples where maximum VCG sells only one item. extend it to $k = 2,3,... ,m$ items.
    \item
    Call an case where only one item is sold $VCG_1, VCG_k$ for k items where $k = 2,3,... ,m.$ How to compute them?
    \item
    Compare maximum $VCG$ revenue to $VCG_1, VCG_2,... ,VCG_k$. Derive the best bound you may obtain.
    \item
    Design a truthful mechanism that derives a best bound in revenue against that of $VCG_1$.
    Compare your revenue against that of $VCG_1$. Extend to $VCG_k$, $k = 1,2,... ,m$.
\end{enumerate}

\begin{answer}
\begin{itemize}
\item Question 2: \\
For the case k = 1, we have example :
$$ N = \begin{Bmatrix}
  11, 7, 3, 2, 1
\end{Bmatrix}, M = \begin{Bmatrix}
  5, 1
\end{Bmatrix}  $$
where we get the maximazation at $VCG(1,1) = 35$ \\

It can be extended by adding $v_1, q_1$ to to the vertors every time:\\
For example, if k = 2 ,  $N = \begin{Bmatrix}
  v_1, 11, 7, 3, 2, 1
\end{Bmatrix}, M = \begin{Bmatrix}
  q_1, 5, 1
\end{Bmatrix}$ where $v_1 > 11$ and $q_1 > 5$.

\item Question 3: \\
To extend the conclusion, for maximum VCG, we have $$VCG(i, j_i) = VCG(i-1, j_{i-1}) + q_{j_i}[iv_{i+1} - (i-1)v_i]$$
If it sells only k items, we have $$VCG(k+1, j_{k+1}) = VCG(k, j_k) + q_{j_{k+1}}[(k+1)v_{k+2} - kv_{k+1}] < VCG(k, j_k)$$
Which gives the condition : \begin{itemize}
  \item $iv_{i+1} - (i-1)v_i > 0 $ for $i <= k$
  \item $(k+1)v_{k+2} - kv_{k+1} < 0 $ if there are more than k items
\end{itemize}

\end{itemize}
\end{answer}

\end{document}
