\documentclass{article}
\usepackage{blindtext}
\usepackage[utf8]{inputenc}

\usepackage{amsthm, amsmath, amssymb}
\usepackage{geometry, setspace, graphicx, enumerate}
\usepackage{listings}
\usepackage[usenames, dvipsnames]{color}
\usepackage{booktabs}

\DeclareMathOperator*{\argmax}{arg\,max}
\DeclareMathOperator*{\argmin}{arg\,min}

\newenvironment{answer}{\par\color{ForestGreen}}{\par}

\title{Summary of L4 -- Linear Utility Market Equilibrium}
\author{Guoxin SUI}
\date{2017 Fall}

\begin{document}

\maketitle

\section{First Half}

\begin{enumerate}
    \item
    Find a mixed strategy Nash equilibrium using Lemke-Howson algorithm for the game below:
    \begin{table}[h!]
      \centering
      \begin{tabular}[t]{ccc}
        \toprule
        events & opera & football \\
        \midrule
        opera & 4,2 & 3,1 \\
        \hline
        football & 1,3 & 2,4 \\
        \bottomrule
     \end{tabular}
   \end{table}
\begin{answer}
    Initiate LH Algorithm:

    $$A = \begin{pmatrix}
          4 & 3 \\
          1 & 2
        \end{pmatrix},
    B = \begin{pmatrix}
          2 & 1 \\
          3 & 4
        \end{pmatrix}, L(x)=\left\{1,2\right\}, L(y)=\left\{3,4\right\}$$
\begin{table}[!htb]
  \begin{answer}
\begin{minipage}[t]{.5\textwidth}
\centering
    \begin{tabular}[t]{cccccc}
      \toprule
      P & x1 & x2 & t3 & t4 & = \\
      \midrule
      3 & 2 & 3 & 1 & 0 & 1 \\
      \hline
      4 & 1 & 4 & 0 & 1 & 1 \\
      \bottomrule
    \end{tabular}
\end{minipage}
\begin{minipage}[t]{0.5\textwidth}
\centering
    \begin{tabular}[t]{llllll}
      \toprule
      Q & r1 & r2 & y3 & y4 & = \\
      \midrule
      1 & 1 & 0 & 4 & 3 & 1 \\
      \hline
      2 & 0 & 1 & 1 & 2 & 1 \\
      \bottomrule
    \end{tabular}
\end{minipage}
\end{answer}
\end{table}

Second step:

$$L(x)=\left\{2,3\right\}, L(y)=\left\{3,4\right\}$$
\begin{table}[!htb]
\begin{answer}
\begin{minipage}[t]{.5\textwidth}
\centering
\begin{tabular}[t]{cccccc}
  \toprule
  P & x1 & x2 & t3 & t4 & = \\
  \midrule
  3 & 2 & 3 & 1 & 0 & 1 \\
  \hline
  4 & 0 & 5 & -1 & 2 & 1 \\
  \bottomrule
\end{tabular}
\end{minipage}
\begin{minipage}[t]{0.5\textwidth}
\centering
\begin{tabular}[t]{llllll}
  \toprule
  Q & r1 & r2 & y3 & y4 & = \\
  \midrule
  1 & 1 & 0 & 4 & 3 & 1 \\
  \hline
  2 & 0 & 1 & 1 & 2 & 1 \\
  \bottomrule
\end{tabular}
\end{minipage}
\end{answer}
\end{table}

Third step:

$$L(x)=\left\{2,3\right\}, L(y)=\left\{1,4\right\}$$
\begin{table}[!htb]
\begin{answer}
\begin{minipage}[t]{.5\textwidth}
\centering
\begin{tabular}[t]{cccccc}
  \toprule
  P & x1 & x2 & t3 & t4 & = \\
  \midrule
  3 & 2 & 3 & 1 & 0 & 1 \\
  \hline
  4 & 0 & 5 & -1 & 2 & 1 \\
  \bottomrule
\end{tabular}
\end{minipage}
\begin{minipage}[t]{0.5\textwidth}
\centering
\begin{tabular}[t]{llllll}
  \toprule
  Q & r1 & r2 & y3 & y4 & = \\
  \midrule
  1 & 1 & 0 & 4 & 3 & 1 \\
  \hline
  2 & -1 & 4 & 0 &5 & 3 \\
  \bottomrule
\end{tabular}
\end{minipage}
\end{answer}
\end{table}

So we have $x_1^*= 1, x_2^*= 0, y_1^*= 1, y_2^*= 0$

\end{answer}
    \item
    Find a mixed strategy Nash equilibrium using Linear Programming for the game below. Also compare your solution with one obtained using Lemke-Howson algorithm.
    \begin{table}[h!]
      \centering
      \begin{tabular}{ccc}
        \toprule
        events & opera & football\\
        \midrule
        opera & 4,-4 & -3,3\\
        \hline
        football & -1,1 & 2,-2\\
        \bottomrule
     \end{tabular}
    \end{table}

    \begin{answer}
        We have:

        $$A = \begin{pmatrix}
              4 & -3 \\
              -1 & 2
            \end{pmatrix}$$

        Set the max value as v :

       $$
        \begin{cases}
          4x_1 - x_2 \geq v \\
          -3x_1 + 2x_2 \geq v
       \end{cases}
        \rightarrow
        \begin{cases}
          4x_1/v - x_2/v \geq 1 \\
          -3x_1/v + 2x_2/v \geq 1
       \end{cases}
       $$

       Suppose $p_1 = x_1/v, p_2 = x_2 /v \rightarrow p_1 + p_2 = 1/v$

       Minimize $1/v \rightarrow
       \begin{cases}
         x_1 = 3/10 \\
         x_2 = 7/10 \\
         v = 1/2
      \end{cases}
       $

       Set the min value as w :

      $$
       \begin{cases}
         4y_1 - 3y_2 \leq w \\
         -y_1 + 2x_2 \leq w
      \end{cases}
       \rightarrow
       \begin{cases}
         4y_1/w - 3y_2/w \leq 1 \\
         -y_1/w + 2x_2/w \leq 1
      \end{cases}
      $$

      Suppose $q_1 = y_1/w, q_2 = y_2 /w \rightarrow q_1 + q_2 = 1/w$

      Minimize $1/w \rightarrow
      \begin{cases}
        y_1 = 1/2 \\
        y_2 = 1/2 \\
        w = 1/2
     \end{cases}
      $

    \end{answer}

    \item
    Implement Lemke-Howson Algorithm (and to solve the above problem, using Pythen or Java). Test it using the above problems.

    \begin{answer}
      The python file is in attachment. We get the same result. The algorithm is also suitable for lager matrix.
    \end{answer}

\end{enumerate}

\section{Second Half}
Let f (·) be continuous. Is there always a fixed point in the following? Prove it or give a counter example.

\begin{enumerate}
    \item
    Continuous Fixed Point $f (x) : [0, 1] \rightarrow [0, 4], f (0) = 1, f(1) = 3.$

    \begin{answer}
      No.

      Couter example: $f(x) = 2x+1$
    \end{answer}

    \item
    Continuous Fixed Point $f (x) : [0, 1] \rightarrow [0, 1], f (0) > 0, f(1) < 1.$

    \begin{answer}
      Yes.

      Construct a function $g(x) = f(x) - x$, then we have $g(0)\le0, g(1)\le0$.

      So there is a point $x^* \subset (0,1)$ that $g(x^*) = 0$ since g is continuous.

      $x^* $is the fixed point.
    \end{answer}

    \item
    Continuous Fixed Point $f (x ) : [0, 1] \rightarrow [1/3, 1/2]$

    \begin{answer}
      Yes.

      Construct a function $g(x) = f(x) - x$, then we have $g(x) \subset (-1/2, 1/2) $.

      So there is a point $x^* \subset (0,1)$ that $g(x^*) = 0$ since g is continuous.

      $x^* $is the fixed point.
    \end{answer}

    \item
    Continuous Fixed Point $f (x ) : [0, 1) \rightarrow  [0, 1).$

    \begin{answer}
      Yes.

      Construct a function $g(x) = f(x) - x$, then we have $g(x) \subset (-1, 1) $.

      So there is a point $x^* \subset (0,1)$ that $g(x^*) = 0$ since g is continuous.

      $x^* $is the fixed point.
    \end{answer}

    \item
    Continuous Fixed Point $f (x ) : [0, \infty) \rightarrow  [0, \infty)$

    \begin{answer}
      Yes.

      Construct a function $g(x) = f(x) - x$, then we have $g(x) \subset (-\infty, \infty) $.

      So there is a point $x^* \subset (0,\infty)$ that $g(x^*) = 0$ since g is continuous.

      $x^* $is the fixed point.
    \end{answer}
\end{enumerate}

\end{document}
