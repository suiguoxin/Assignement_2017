\documentclass{article}
\usepackage{blindtext}
\usepackage[utf8]{inputenc}

\usepackage{amsthm, amsmath, amssymb}
\usepackage{geometry, setspace, graphicx, enumerate}
\usepackage{listings}
\usepackage[usenames, dvipsnames]{color}
\usepackage{booktabs}

\DeclareMathOperator*{\argmax}{arg\,max}
\DeclareMathOperator*{\argmin}{arg\,min}

\newenvironment{answer}{\par\color{ForestGreen}}{\par}
\setlength\parindent{0pt}

\title{Midterm}
\author{Guoxin SUI}
\date{\today}

\begin{document}
\maketitle
\section{Problem 1}
\begin{answer}
  Assume that the total size of the cake is 1. Denote the physical size of the pieces by $x_1, x_2, x_3$. Since $x_1 + x_2 + x_3 = 1 $and all $x_i \geq 0$, the solution space is just a triangle. Assign the three people as A,B,C, then we can a traingle where each elementary triangle is an ABC triangle, the A, B, C present the "ownership" of the vertice. A similar triangulation of finer mesh can also be labelled in this way.
\end{answer}
\section{Problem 2}
\begin{answer}
    Initiate LH Algorithm:

    $$A = \begin{pmatrix}
          3 & 2 \\
          1 & 3
        \end{pmatrix},
    B = \begin{pmatrix}
          1 & 5 \\
          4 & 2
        \end{pmatrix},$$
\begin{table}[!htb]
  \begin{answer}
\begin{minipage}[t]{.5\textwidth}
\centering
    \begin{tabular}[t]{cccccc}
      \toprule
      P & x1 & x2 & t3 & t4 & = \\
      \midrule
      3 & 1 & 4 & 1 & 0 & 1 \\
      \hline
      4 & 5 & 2 & 0 & 1 & 1 \\
      \bottomrule
    \end{tabular}
\end{minipage}
\begin{minipage}[t]{0.5\textwidth}
\centering
    \begin{tabular}[t]{llllll}
      \toprule
      Q & r1 & r2 & y3 & y4 & = \\
      \midrule
      1 & 1 & 0 & 3 & 2 & 1 \\
      \hline
      2 & 0 & 1 & 1 & 3 & 1 \\
      \bottomrule
    \end{tabular}
\end{minipage}
\end{answer}
\end{table}
$$L(x)=\left\{1,2\right\}, L(y)=\left\{3,4\right\}$$

Second step:

\begin{table}[!htb]
\begin{answer}
\begin{minipage}[t]{.5\textwidth}
\centering
\begin{tabular}[t]{cccccc}
  \toprule
  P & x1 & x2 & t3 & t4 & = \\
  \midrule
  3 & 1 & 4 & 1 & 0 & 1 \\
  \hline
  4 & 9 & 0 & -1 & 2 & 1 \\
  \bottomrule
\end{tabular}
\end{minipage}
\begin{minipage}[t]{0.5\textwidth}
\centering
\begin{tabular}[t]{llllll}
  \toprule
  Q & r1 & r2 & y3 & y4 & = \\
  \midrule
  1 & 1 & 0 & 3 & 2 & 1 \\
  \hline
  2 & 0 & 1 & 1 & 3 & 1 \\
  \bottomrule
\end{tabular}
\end{minipage}
\end{answer}
\end{table}
$$L(x)=\left\{1,3\right\}, L(y)=\left\{3,4\right\}$$

Third step:

\begin{table}[!htb]
\begin{answer}
\begin{minipage}[t]{.5\textwidth}
\centering
\begin{tabular}[t]{cccccc}
  \toprule
  P & x1 & x2 & t3 & t4 & = \\
  \midrule
  3 & 1 & 4 & 1 & 0 & 1 \\
  \hline
  4 & 9 & 0 & -1 & 2 & 1 \\
  \bottomrule
\end{tabular}
\end{minipage}
\begin{minipage}[t]{0.5\textwidth}
\centering
\begin{tabular}[t]{llllll}
  \toprule
  Q & r1 & r2 & y3 & y4 & = \\
  \midrule
  1 & 1 & 0 & 3 & 2 & 1 \\
  \hline
  2 & -1 & 3 & 0 & 7 & 2 \\
  \bottomrule
\end{tabular}
\end{minipage}
\end{answer}
\end{table}
$$L(x)=\left\{1,3\right\}, L(y)=\left\{1,4\right\}$$

Fourth step:

\begin{table}[!htb]
\begin{answer}
\begin{minipage}[t]{.5\textwidth}
\centering
\begin{tabular}[t]{cccccc}
  \toprule
  P & x1 & x2 & t3 & t4 & = \\
  \midrule
  3 & 0 & 18 & 5 & -1 & 4 \\
  \hline
  4 & 9 & 0 & -1 & 2 & 1 \\
  \bottomrule
\end{tabular}
\end{minipage}
\begin{minipage}[t]{0.5\textwidth}
\centering
\begin{tabular}[t]{llllll}
  \toprule
  Q & r1 & r2 & y3 & y4 & = \\
  \midrule
  1 & 1 & 0 & 3 & 2 & 1 \\
  \hline
  2 & -1 & 3 & 0 & 7 & 2 \\
  \bottomrule
\end{tabular}
\end{minipage}
\end{answer}
\end{table}
$$L(x)=\left\{3,4\right\}, L(y)=\left\{1,4\right\}$$

Fifth step:

\begin{table}[!htb]
\begin{answer}
\begin{minipage}[t]{.5\textwidth}
\centering
\begin{tabular}[t]{cccccc}
  \toprule
  P & x1 & x2 & t3 & t4 & = \\
  \midrule
  3 & 0 & 18 & 5 & -1 & 4 \\
  \hline
  4 & 9 & 0 & -1 & 2 & 1 \\
  \bottomrule
\end{tabular}
\end{minipage}
\begin{minipage}[t]{0.5\textwidth}
\centering
\begin{tabular}[t]{llllll}
  \toprule
  Q & r1 & r2 & y3 & y4 & = \\
  \midrule
  1 & 3 & -2 & 7 & 0 & 1 \\
  \hline
  2 & -1 & 3 & 0 & 7 & 2 \\
  \bottomrule
\end{tabular}
\end{minipage}
\end{answer}
\end{table}
$$L(x)=\left\{3,4\right\}, L(y)=\left\{1,2\right\}$$

So we have a mixed strategy equilibrium.
\end{answer}
\section{Problem 3}
\begin{answer}
  \paragraph{(a)}
  \begin{itemize}
    \item Budget constraint :

    $px^i \leq pw^i $
    \item Individual optimality :

    $x^{i*} \in argmax \{u_i(x^i) : px^i \leq pw^{iT},x^i \geq 0 \}$,
    where
    $u^1 = \begin{bmatrix} 3, 0, 0, 0, 0 \end{bmatrix}^T$
    $u^2 = \begin{bmatrix} 0, 4, 2, 0, 0 \end{bmatrix}^T$
    $u^3 = \begin{bmatrix} 0, 2, 1, 0, 0 \end{bmatrix}^T$

    \item Market clearance:

    $\sum_{i\in M} x^i \leq \sum_{i\in M} w^{iT}$
  \end{itemize}

  To get the maximum utility, the agent will put his money on the good where he get most utility with every unit of money since his budget constraints to the equation $px^i \leq pw^i $. So if $x_j^{i*} >0$, that means $\frac{u_j^i}{p_j}$ is the maximum.

  \paragraph{(b)}
  \begin{itemize}
    \item Normalization:
    \begin{itemize}
      \item Everything is owned by someone: Condition fulfilled.
      \item Everything is liked by someone: Elimite the $5_{th}$ good.
      Then we have 4 goods at the market $ N = \begin{Bmatrix} 1,2,3,4 \end{Bmatrix}$.
      The initial endowment of the agents
      $ w^1 = \begin{pmatrix} 0, 2, 0, 1 \end{pmatrix},
      w^2 = \begin{pmatrix} 0, 2, 1, 0 \end{pmatrix},
      w^3 = \begin{pmatrix} 1, 0, 0, 3 \end{pmatrix}$,
      \item Normalization: For the $2_{nd}$ and $4_{th}$ good, divide by total amount.
      The initial endowment of the agents
      $ w^1 = \begin{pmatrix} 0, 1/2, 0, 1/4 \end{pmatrix},
      w^2 = \begin{pmatrix} 0, 1/2, 1, 0 \end{pmatrix},
      w^3 = \begin{pmatrix} 1, 0, 0, 3/4 \end{pmatrix}$,
    \end{itemize}

    \item Atomization:
    \begin{itemize}
      \item Every agent owns one item : replace agent 1, 2, 3 by $ 1_1, 1_2, 2_1, 2_2, 3_1, 3_2$,, then we have
      $6$ markets agents $ M = \begin{Bmatrix} 1_1, 1_2, 2_1, 2_2, 3_1, 3_2 \end{Bmatrix}$,
      where
     $\begin{cases}
        u^{1_1} = \begin{bmatrix} 3, 0, 0, 0\end{bmatrix}^T \\
        u^{1_2} = \begin{bmatrix} 3, 0, 0, 0\end{bmatrix}^T \\
        u^{2_1} = \begin{bmatrix} 0, 4, 2, 0\end{bmatrix}^T \\
        u^{2_2} = \begin{bmatrix} 0, 4, 2, 0\end{bmatrix}^T \\
        u^{3_1} = \begin{bmatrix} 0, 2, 0, 1\end{bmatrix}^T \\
        u^{3_2} = \begin{bmatrix} 0, 2, 0, 1\end{bmatrix}^T
      \end{cases} $

      The initial endowment of the agents
      $\begin{cases}
         w^{1_1} = \begin{pmatrix} 0, 2, 0, 0 \end{pmatrix} \\
         w^{1_2} = \begin{pmatrix} 0, 0, 0, 1 \end{pmatrix} \\
         w^{2_1} = \begin{pmatrix} 0, 2, 0, 0 \end{pmatrix} \\
         w^{2_2} = \begin{pmatrix} 0, 0, 1, 0 \end{pmatrix} \\
         w^{3_1} = \begin{pmatrix} 1, 0, 0, 0 \end{pmatrix} \\
         w^{3_2} = \begin{pmatrix} 0, 0, 0, 3 \end{pmatrix}
       \end{cases} $

      \item Every item is owned by one agent: Rename the same type of items own by different agents and equalize the utilities by an agent on them, then we have
      6 goods at the market $ N = \begin{Bmatrix} 1,2_1, 2_2, 3_1, 3_2, 4 \end{Bmatrix}$.
      $\begin{cases}
         w^{1_1} = \begin{pmatrix} 0, 2, 0, 0, 0, 0 \end{pmatrix} \\
         w^{1_2} = \begin{pmatrix} 0, 0, 0, 0, 1, 0 \end{pmatrix} \\
         w^{2_1} = \begin{pmatrix} 0, 0, 2, 0, 0, 0 \end{pmatrix} \\
         w^{2_2} = \begin{pmatrix} 0, 0, 0, 1, 0, 0 \end{pmatrix} \\
         w^{3_1} = \begin{pmatrix} 1, 0, 0, 0, 0, 0 \end{pmatrix} \\
         w^{3_2} = \begin{pmatrix} 0, 0, 0, 0, 0, 3 \end{pmatrix}
       \end{cases} $
       here
      $\begin{cases}
         u^{1_1} = \begin{bmatrix} 3, 0, 0, 0, 0, 0\end{bmatrix}^T \\
         u^{1_2} = \begin{bmatrix} 3, 0, 0, 0, 0, 0\end{bmatrix}^T \\
         u^{2_1} = \begin{bmatrix} 0, 4, 4, 2, 0, 0\end{bmatrix}^T \\
         u^{2_2} = \begin{bmatrix} 0, 4, 4, 2, 0, 0\end{bmatrix}^T \\
         u^{3_1} = \begin{bmatrix} 0, 2, 2, 0, 1, 1\end{bmatrix}^T \\
         u^{3_2} = \begin{bmatrix} 0, 2, 2, 0, 1, 1\end{bmatrix}^T
       \end{cases} $

      \item Normalization revision: Divide by its size, we get
      $\begin{cases}
         w^{1_1} = \begin{pmatrix} 0, 1, 0, 0, 0, 0 \end{pmatrix} \\
         w^{1_2} = \begin{pmatrix} 0, 0, 0, 0, 1, 0 \end{pmatrix} \\
         w^{2_1} = \begin{pmatrix} 0, 0, 1, 0, 0, 0 \end{pmatrix} \\
         w^{2_2} = \begin{pmatrix} 0, 0, 0, 1, 0, 0 \end{pmatrix} \\
         w^{3_1} = \begin{pmatrix} 1, 0, 0, 0, 0, 0 \end{pmatrix} \\
         w^{3_2} = \begin{pmatrix} 0, 0, 0, 0, 0, 1 \end{pmatrix}
       \end{cases} $
    \end{itemize}


    \item Since everything is liked by someone (The goods that no one likes are eliminated), and an edge from i to j means agent i likes item j, there is non-zero indegree.
  \end{itemize}

  \paragraph{(c)}
  \hfill \break
  In "Linear Utility Market", the goods are initialely distributed to the agents, but they don't have endowment of money.

  In "Fisher Market", there is a seller who owns all the goods, but the agents have endowment of money to buy the goods.


  These differences lead to future differences in market clearance and budget constraints.
\end{answer}

\section{Problem 1}
\begin{answer}
  \begin{itemize}
    \item Bidder 1 get q1 and bidder 2 get q2, \\
    $p_1 = v_2q_1 + v_3q_2 - v_2q_2 = 26$, \\
    $p_2 = v_1q_1 + v_3q_2 - v_1q_1 = 12$
    \item We have
    $\begin{cases}
      u_1(q_1) \geq 0 \\
      u_1(q_1) \geq u_1(q_2) \\
      u_2(q_2) \geq 0 \\
      u_2(q_2) \geq u_2(q_1) \\
      u_3(q_1) \leq 0 \\
      u_3(q_2) \leq 0
    \end{cases}
    \Rightarrow
    \begin{cases}
      v_1q_1 - p_1 \geq 0 \\
      v_1q_1 - p_1 \geq v_1q_2 - p_2 \\
      v_2q_2 - p_2 \geq 0 \\
      v_2q_2 - p_2 \geq v_2q_1 - p_1 \\
      v_3q_1 - p_1 \leq 0 \\
      v_3q_2 - p_2 \leq 0
    \end{cases}
    \Rightarrow
    \begin{cases}
      14 \leq  p_1 \leq 36 \\
      6 \leq  p_2 \leq 14 \\
      14 \leq  p_1 - p_2 \leq 18
    \end{cases}$
    \item Truthful bidding under the GSP protocol is optimal for the buyers since no bidder would change its bid to improve its utility.
    In this case, if all start by bidding their true vale, this is already a Nash equilibrium.
  \end{itemize}
\end{answer}

\end{document}
